\capitulo{1}{Introducción}

A día de hoy uno de los mayores gastos a los que se puede enfrentar una empresa, y que suponen una gran pérdida de tiempo y dinero, se encuentra en el mantenimiento de las fábricas. Las que más mantenimiento requieren son aquellas con una cadena de producción, o que poseen diferentes máquinas y/o herramientas de la cuales se hacen un uso muy intensivo. Actualmente hay muy pocas opciones para llegar a la solución de este problema de alto mantenimiento económico, lo más común es tener a uno o varios operarios cualificados en cada una de las fábricas de la empresa para resolver los contratiempos en el menor tiempo posible. Pero esto tiene un inconveniente, el gran desembolso económico que supone contar con varios operarios por fábrica, lo cual no es asumible. El cual muchas empresas pueden no ser capaces de afrontar.

La solución propuesta por este TFG realizado en conjunto con el Instituto Tecnológico de Castilla y León (ITCL) \cite{itcl:latex} es crear una aplicación para las HoloLens 2 \cite{microsoft:hololens2}, dispositivo de realidad mixta desarrollado por Microsoft orientado a aumentar la productividad de las empresas, en la que un técnico pueda guiar y dar instrucciones a cualquier usuario que se encuentre disponible en la fábrica, aunque tenga poca o nula formación sobre mantenimiento, para así solucionar cualquier imprevisto que surja. Gracias a esto se puede conseguir reducir el número de técnicos por fábrica, lo cual supondría reducir considerablemente el gasto económico en términos de personal. Además, una gran ventaja que ofrece este sistema es que puede ser escalado a nivel global, significando esto que los técnicos puedan ofrecer sus conocimientos a cualquier parte del planeta donde se disponga de una conexión a internet.

Una vez completado el prototipo de la aplicación se esperan implementar funcionalidades como compartir documentos en realidad aumentada, reconocimiento de objetos, interacción con piezas o maquinaria en 3D.

\section{Estructura de la memoria}

La memoria sigue la siguiente estructura:
\begin{description}
	\item[Introducción:] breve descripción del problema a resolver y la solución planteada. Estructura de la memoria y los Materiales adjuntos.
	\item[Objetivos del proyecto:] requisitos necesarios para cumplir los objetivos del proyecto.
	\item[Conceptos teóricos:] explicación de los conceptos teóricos usados a lo largo de la memoria.
	\item[Técnicas y herramientas:] metodología y herramientas utilizadas durante el desarrollo.
	\item[Aspectos relevantes del desarrollo:] explicación de los aspectos más relevantes del desarrollo.
	\item[Trabajos relacionados] pequeño resumen comentado de los trabajos y proyectos ya realizados relacionados con este campo.
	\item[Conclusiones y líneas de trabajo futuras] conclusiones obtenidas tras la finalización del proyecto y posibilidades de expansión y mejora.
\end{description}

\section{Anexos}
\begin{description}
	\item[Plan de proyecto software:] planificación de la estructura del proyecto y la viabilidad económica.
	\item[Especificación de requisitos:] requisitos marcados por los objetivos del proyecto.
	\item[Especificación de diseño] diseño del software y sus diagramas correspondientes.
	\item[Manual del programador:] recoge todos los aspectos necesarios para trabajar con los ficheros del proyecto.
	\item[Manual del usuario:] guía del usuario para la aplicación.
	
\end{description}
\section{Materiales Adjuntos}

Los materiales adjuntos a este TFG son:
\begin{description}
	\item[Documentación:] Documentos en formato digital de la memoria y los anexos.
	\item[\textit{Builds:}] Carpeta con los ejecutables de las aplicaciones. 
	\item[Fuentes:] Directorio con el proyecto de Unity.
	\item[Fuentes de documentación:] Directorio del proyecto de \LaTeX.
	\item[Vídeos:] Vídeos del producto final.
\end{description}
