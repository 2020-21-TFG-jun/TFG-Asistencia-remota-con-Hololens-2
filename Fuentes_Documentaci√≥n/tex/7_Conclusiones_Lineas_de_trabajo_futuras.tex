\capitulo{7}{Conclusiones y Líneas de trabajo futuras}

\section{Conclusiones}

Tras la finalización de este proyecto se puede concluir que todos los objetivos principales han sido completados satisfactoriamente. Dando lugar a una aplicación prototipo que permite la comunicación mediante vídeo y audio del dispositivo de realidad mixta HoloLens 2 y un ordenador. Otra funcionalidad presente en el resultado final es la capacidad de asistir remotamente a través de los indicadores generados en realidad aumentada.

En el proyecto se han conseguido incluir parte de los objetivos secundarios propuestos. Se ha implementado funcionalidades básicas que caracterizan las aplicaciones de comunicación por voz y vídeo. Puedes silenciar tu micrófono, apagar la webcam, llamar, colgar, etc. También se ha desarrollado una interfaz gráfica basada en botones para poder acceder a estas funcionalidades.

Al comienzo del proyecto tenía cierta preocupación puesto que tenía un límite establecido de 300 horas y nunca había trabajado ninguna de las herramientas utilizadas. Esta cantidad de horas resultó ser muy justa, ya que terminé el prototipo en la última semana del contrato de prácticas. Tuve que invertir más horas de las establecidas en el TFG debido a que posteriormente a la finalización del contrato en ITCL le dediqué una cantidad bastante considerable de horas para trabajar en la documentación. Comencé completamente de 0, sin ningún conocimiento de Unity, C\# ni de las HoloLens 2, por lo que al comienzo del proyecto no podía realizar una estimación clara de todo el trabajo a realizar.

Trabajar con Unity fue relativamente sencillo, las características principales de este motor gráfico son fáciles de aprender y la interfaz es bastante intuitiva. Afirmación que no puedo realizar con las HoloLens 2. A lo largo de la realización TFG, encontré una cantidad considerable de bugs relacionados con el software de las gafas. El rendimiento final de la aplicación también se vio disminuido debido a limitaciones de hardware.

Personalmente estoy satisfecho con el resultado final, por que se han cumplido todos objetivos principales y parte de los secundarios. Durante el proyecto he empleado muchas herramientas y conocimientos adquiridos en el grado, que me ha sido de gran utilidad durante ciertas fases del proyecto. También he adquirido nueva experiencia usando herramientas como Unity, \LaTeX, Jira... 

Para terminar, quiero dar las gracias a ITCL, puesto que con su colaboración, he tenido la oportunidad de trabajar con una tecnología a la que muy pocos tienen acceso debido a su exclusividad y elevado coste. También quiero agradecer a Ángel Arroyo, Nuño Basurto y Alejandro Langarica por ser mis tutores de este TFG.

\section{Líneas de trabajo futuras}

Las principales mejoras a incluir a este prototipo son:
\begin{itemize}
	\item Evolucionar el método actual utilizado para realizar la conexión entre los 2 dispositivos. Obtener como resultado final una comunicación a través de internet de forma global.
	\item Implementar un sistema de reconocimiento de objetos combinando el proyecto actual con el servicio de Microsoft Azure.
	\item Crear una interfaz previa a la llamada donde puedas agregar contactos y comunicarte con ellos mediante texto.
	\item Cargar modelos 3D en las HoloLens 2 para visualizarlos en realidad aumentada.
	\item Enviar un documento en formato .pdf desde el ordenador para generarlo en realidad aumentada, haciendo así uso de él.
\end{itemize}



